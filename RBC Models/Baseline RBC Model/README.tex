\documentclass[12pt,english]{extarticle}
\usepackage[english]{babel}
\usepackage[utf8]{inputenc}
\usepackage[UKenglish,nodayofweek]{datetime}
\newdateformat{UKvardate}{
	\THEDAY \monthname[\THEMONTH], \THEYEAR
}

\usepackage[autostyle]{csquotes}
\MakeOuterQuote{"} %fixes quotation marks
\usepackage{fancyhdr}
\usepackage{graphicx}
\usepackage{geometry}
\usepackage{comment}
\usepackage{longtable}
\usepackage{booktabs}
\geometry{
	a4paper,
	left=2.5cm,
	right=2.5cm,
	top=3cm,
	bottom=3cm
}
\usepackage{setspace}
\onehalfspacing
\usepackage{multirow}
\usepackage{pdflscape}
\usepackage{float}
\usepackage{amsmath}
\usepackage{amsfonts}
\usepackage{amssymb}
\usepackage{amsthm}
\usepackage{bm}
\usepackage{bbm}
\usepackage{euscript} %for subspace S notation with \EuScript{}

%\input{biblatex-aer}%biblatex-aer.tex needs to be TeX document folder
%\addbibresource{references.bib}

\DeclareUnicodeCharacter{0301}{\'{e}}
\DeclareUnicodeCharacter{0301}{\'{i}}

\usepackage[colorlinks=true,linkcolor=red,urlcolor=black,citecolor=blue,anchorcolor=red]{hyperref}
%\usepackage[hidelinks]{hyperref}
\usepackage{xurl}

\usepackage{caption}
\usepackage{subcaption}

\newtheorem{fact}{Fact}[]

% Set Var, AVar, plim and E characters:
\DeclareMathOperator*{\Var}{Var}
\DeclareMathOperator*{\Cov}{Cov}
\DeclareMathOperator*{\AVar}{AVar}
\DeclareMathOperator*{\plim}{plim}
\newcommand{\E}{\mathbb{E}}

%\date{1 April 2023} % Update date here

\title{\textbf{A Baseline RBC Model}}
\author{David Murakami}
\date{\today}

\begin{document}
	\maketitle
	
	\section*{Ramsey social planner problem}
	\label{sec: baseline RBC model}
	Consider a Hansen-style RBC model.Final good firms use the production function $Y_t = A_tK_{t-1}^\alpha L_t^{1-\alpha}$. The representative household utility function is $u(C_t,L_t) = \ln C_t - \chi L^{1+\nu^{-1}}/(1+\nu^{-1})$. Set $\alpha=0.3$, $\nu=2$, $\chi=4.5$, $\beta=0.99$, $\delta=0.025$, $\rho_a=0.95$, and $\sigma_a=0.01$.
	
	As there are no distortions, we can solve the model from the perspective of the Ramsey social planner who aims to
	\begin{equation*}
		\max_{\{C_t,L_t,K_t\}} \, \E_t \sum_{i=0}^\infty \beta^i \left(\ln C_{t+i} - \chi \frac{L_{t+i}^{1+\frac{1}{\nu}}}{1+\frac{1}{\nu}} \right),
	\end{equation*}
	subject to 
	\begin{align}
		\label{eq: aggregate resource constraint RBC}
		Y_t &= C_t + I_t, \\
		\label{eq: law of motion of capital RBC}
		K_t &= I_t + (1-\delta)K_{t-1}, \\
		\label{eq: production function RBC}
		Y_t &= A_tK_{t-1}^\alpha L_t^{1-\alpha}, \\
		\label{eq: TFP process RBC}
		\ln A_t &= \rho_a \ln A_{t-1} + \varepsilon_t^a, \quad \varepsilon_t^a \overset{IID}{\sim} \mathcal{N}(0,\sigma_a^2).
	\end{align}
	Solving the problem yields the following first-order conditions:
	\begin{align*}
		\frac{1}{C_t} &= \beta \E_t\left[\frac{1}{C_{t+1}}\left(\alpha\frac{Y_{t+1}}{K_t} + 1 - \delta\right)\right], \\
		\chi L_t^{\frac{1}{\nu}} &= \frac{(1-\alpha)Y_t}{C_tL_t}.
	\end{align*}
	
	Define the marginal value of an additional unit of capital in period $t+1$ as
	\begin{equation}
		\label{eq: capital return RBC}
		R_{t+1} = \alpha\frac{Y_{t+1}}{K_t} + 1 - \delta,
	\end{equation}
	and define the marginal product of labour as the real wage,
	\begin{equation}
		\label{eq: wages RBC}
		W_t = (1-\alpha)\frac{Y_t}{L_t}
	\end{equation}
	Use these to write the FOCs as:
	\begin{align}
		\label{eq: consumption euler equation RBC}
		\frac{1}{C_t} &= \beta \E_t\left[\frac{R_{t+1}}{C_{t+1}}\right], \\
		\label{eq: intratemporal euler equation RBC}
		W_t &= \chi L_t^\frac{1}{\nu}C_t.
	\end{align}
	
	\paragraph{Centralised equilibrium.} The equilibrium is a set of prices, $R_t$ and $W_t$; allocations, $Y_t$, $C_t$, $L_t$, $I_t$, and $K_t$; and productivity, $A_t$, which satisfy eight equations, \eqref{eq: aggregate resource constraint RBC}-\eqref{eq: intratemporal euler equation RBC}.
	
	\paragraph{Steady state.} Steady state quantities are found by first using the fact that $A=1$, and from \eqref{eq: capital return RBC} and \eqref{eq: consumption euler equation RBC} we have:
	\begin{equation*}
		\begin{split}
			\frac{1}{\beta} &= \alpha\left(\frac{K}{L}\right)^{\alpha-1} + 1 - \delta \\
			\implies \frac{K}{L} &= \left(\frac{\alpha}{\beta^{-1}-1+\delta}\right)^{\frac{1}{1-\alpha}}.
		\end{split}
	\end{equation*}
	Then express $W$ in \eqref{eq: wages RBC} as:
	\begin{equation*}
		W = (1-\alpha)\left(\frac{K}{L}\right)^\alpha.
	\end{equation*}
	
	From \eqref{eq: law of motion of capital RBC} we know:
	\begin{equation*}
		I = \delta K.
	\end{equation*}
	Then, combine this with \eqref{eq: aggregate resource constraint RBC} and \eqref{eq: production function RBC} to get the consumption-output ratio:
	\begin{equation*}
		\frac{C}{L} = \left(\frac{K}{L}\right)^\alpha - \delta\left(\frac{K}{L}\right).
	\end{equation*}
	
	Next, turn to the intratemporal Euler equation \eqref{eq: intratemporal euler equation RBC}, and multiply both sides by $L$:
	\begin{equation*}
		\chi L^{\frac{1+\nu}{\nu}} = \frac{(1-\alpha)L}{C}\left(\frac{K}{L}\right)^\alpha,
	\end{equation*}
	and then rearrange to get $C/L$:
	\begin{equation*}
		\frac{C}{L} = \frac{(1-\alpha)\left(\frac{K}{L}\right)^\alpha}{\chi L^{\frac{1+\nu}{\nu}}}.
	\end{equation*}
	Set this equal to the previous expression for $C/L$ to solve for $L$:
	\begin{equation*}
		L = \left\{\frac{(1-\alpha)\left(\frac{K}{L}\right)^\alpha}{\chi\left[\left(\frac{K}{L}\right)^\alpha - \delta\frac{K}{L}\right]}\right\}^{\frac{\nu}{1+\nu}}.
	\end{equation*}
	
	With $L$ in hand, turn back to the capital-labour ratio to obtain $K$:
	\begin{equation*}
		K = \left(\frac{\alpha}{\beta^{-1}-1+\delta}\right)^{\frac{1}{1-\alpha}}L.
	\end{equation*}
	Thus, $I$ and $Y$ can easily be computed, after which $C$ can obtained easily using the aggregate resource constraint \eqref{eq: aggregate resource constraint RBC}.
	
\end{document}